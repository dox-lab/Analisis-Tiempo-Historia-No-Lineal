\chapter*{\centering \large RESUMEN} 
%\addcontentsline{toc}{chapter}{Resumen} % si queremos que aparezca en el índice
\markboth{Resumen}{Resumen} % encabezado
\onehalfspacing
\setlength{\parskip}{1em} % Ajusta el espacio entre párrafos

{
Para realizar el diseño de una edificación, se hace un modelo y un análisis estructural que nos permite predecir el comportamiento de la estructura ante las solicitaciones de servicio, sísmicas, entre otros. Para realizar este análisis, las técnicas ingenieriles realizan demasiadas asunciones que simplifican los cálculos. Sin embargo, no se tiene una forma de medir la veracidad de estos resultados de forma experimental. Los diseños se basan en la experiencia y criterio del ingeniero a cargo, pero este conocimiento puede fallar.

Por este motivo, las normas de construcción peruana nos indican que tenemos que aplicar un sobrediseño para tener un factor de seguridad, lo cual encarece el proyecto. Esta tesis tiene como objetivo desarrollar una herramienta, con interfaz gráfica, de fácil uso e interpretación, en Python, que permita medir la confiabilidad de los resultados del análisis estructural hallados con métodos discretos (exactos). En otras palabras, medir la probabilidad de ocurrencia de los resultados hallados previamente.

Este análisis de confiabilidad estructural se realizará con la librería de Python OpenSeesPy, la cual realizará el análisis estructural; seguidamente se agregará la variabilidad a las propiedades de la estructura y solicitaciones (señales sísmicas, propiedades de concreto, variaciones en construcción, entre otros).
}
