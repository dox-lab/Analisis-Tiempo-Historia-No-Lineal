\chapter{Introducción}

\section{Planteamiento del problema}
Los criterios de diseño, ya sea en concreto armado, en acero o en cualquier otro material, se basan en la normativa de construcción. En esta normativa se presentan ciertos criterios y requisitos mínimos para la construcción de diferentes obras. Estos criterios están sustentados en los resultados experimentales de diferentes ensayos, experiencia pasada y nuevos métodos de análisis computacional. 

Sin embargo, para el caso de diseño sismorresistente, los criterios son difíciles de corroborar y sustentar experimentalmente, ya que los ensayos para este tipo de solicitación no son posibles a nivel real de la estructura debido a la magnitud, los recursos o el tiempo. Por este motivo, las normas de diseño sismorresistente se basan principalmente en la experiencia de anteriores eventos sísmicos.

En el Perú, la norma técnica que rige los criterios de construcción para solicitaciones de sismo es la norma peruana E.030 de Diseño Sismorresistente. Este documento presenta métodos de calificación de estructuras, peligro sísmico por zonas y el comportamiento esperado ante un sismo, entre otras cosas. A pesar de lo indicado en la norma, existe una alta posibilidad de que la respuesta final de la estructura no esté contemplada en el documento. Esto se debe a que los sismos son eventos aleatorios y, como se mencionó anteriormente, la norma se basa en la experiencia de eventos sísmicos pasados. Sin embargo, en el Perú no existen registros de suficientes eventos sísmicos para alcanzar un nivel adecuado de confiabilidad.

Además de este problema, en la norma E.060 de Concreto Armado, el material más usado para la construcción en el mundo, se mencionan propiedades del concreto, como la resistencia a la compresión (\textit{f'c}) y el módulo de Young. Estas propiedades no tienen un valor constante, sino que varían de acuerdo con la calidad de las materias primas. Además de esto, se debe considerar la variabilidad de los métodos de construcción utilizados en la ejecución de la obra. 

Toda esta variabilidad en las propiedades, características, dimensiones, métodos de construcción, modelos de análisis, limitaciones computacionales y la aleatoriedad de los eventos sísmicos produce una desconfianza en los resultados utilizados para el diseño de estructuras. Como contramedida a esta situación, las normas indican la sobrerresistencia de los elementos estructurales. Sin embargo, esta solución, aunque viable y comprobada, encarece el proyecto final tanto a nivel económico como de tiempo. Cabe agregar que esta sobrerresistencia, también considerada como el factor de seguridad, brinda un falso nivel de confianza, ya que, como se mencionó anteriormente, los eventos aleatorios y las propiedades variables hacen que la respuesta de la estructura frente a estas cargas sea incierta, incluso llegando a casos extraordinarios donde la estructura podría colapsar.

La presente propuesta surge a partir de la problemática mencionada y presenta una metodología y herramientas digitales para realizar una medición de la confiabilidad estructural.

\section{Antecedentes y justificación}
La ingeniería sismorresistente es una de las ramas más recientes de la ingeniería civil. Uno de sus objetivos es que las obras civiles no colapsen ante eventos sísmicos y que puedan asegurar el resguardo de la vida (Muñoz, 2020). A lo largo de la historia del Perú, se han creado normas de diseño que buscan cumplir este objetivo, proporcionando parámetros mínimos para garantizar un comportamiento adecuado. 

Una de las primeras recomendaciones para el diseño sismorresistente se oficializó en 1977 en la normativa peruana de construcción. En 1996, se produjo el sismo de Nasca, de magnitud 7.7 Mw, que causó la destrucción de colegios y viviendas en toda la zona. A raíz de este evento, se creó la norma sismorresistente de 1997. Sin embargo, en 2001 ocurrió un sismo en la localidad de Arequipa, con una magnitud de 7.9 Ms, que dejó muchos colegios inutilizables, lo cual evidenció que la norma no contemplaba muchos aspectos de las unidades escolares. Este evento marcó el inicio de la redacción de una nueva norma sismorresistente, publicada en 2003. A lo largo de los años, se publicó una versión de la norma en 2016 y, la actualmente vigente, en 2018.

En cada una de las normas antes mencionadas, los eventos sísmicos han proporcionado información valiosa para mejorar, corregir y reescribir la normativa de diseño sismorresistente. Sin embargo, este enfoque limita mucho la implementación de nuevas metodologías de construcción y diseño. Por este motivo, en este documento se plantea una metodología para realizar simulaciones sísmicas y comparaciones de modelos estructurales. Adicionalmente, se incluye una medición de confiabilidad estructural.

Contar con resultados de confiabilidad permite reducir los factores de seguridad, ya que, al realizar un análisis de confiabilidad, se tiene una mejor seguridad del comportamiento real. Al reducir los factores de seguridad, el proyecto de construcción disminuirá sus costos económicos y aumentará la seguridad de la edificación.

\section{Hipótesis}
Se asume que la deriva de un edificio al aplicar una carga lateral aleatoria está directamente relacionada con dicha carga; por lo tanto, el análisis probabilístico nos dará resultados similares a los de un análisis discreto.

\section{Objetivos}
El objetivo general de la tesis es brindar una herramienta para realizar un análisis de confiabilidad estructural.

Los objetivos específicos del proyecto son:
\begin{itemize}
    \item Verificar la fiabilidad de los resultados hallados por métodos discretos mediante la probabilidad de ocurrencia de estos.
    \item Aplicar técnicas de programación orientada a objetos y simulación aleatoria.
    \item Crear un programa para obtener señales sísmicas sintéticas.
    \item Aplicar técnicas de muestreo para la optimización del análisis de confiabilidad.
\end{itemize}
